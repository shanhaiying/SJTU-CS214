\documentclass[12pt,a4paper]{article}
%\usepackage{ctex}
\usepackage{amsmath,amscd,amsbsy,amssymb,latexsym,url,bm,amsthm}
\usepackage{epsfig,graphicx,subfigure}
\usepackage{enumitem,balance}
\usepackage{wrapfig}
\usepackage{mathrsfs,euscript}
\usepackage[usenames]{xcolor}
\usepackage{hyperref}
\usepackage[vlined,ruled,linesnumbered]{algorithm2e}
\usepackage{array}
\hypersetup{colorlinks=true,linkcolor=black}

\newtheorem{theorem}{Theorem}
\newtheorem{lemma}[theorem]{Lemma}
\newtheorem{proposition}[theorem]{Proposition}
\newtheorem{corollary}[theorem]{Corollary}
\newtheorem{exercise}{Exercise}
\newtheorem*{solution}{Solution}
\newtheorem{definition}{Definition}
\theoremstyle{definition}

\renewcommand{\thefootnote}{\fnsymbol{footnote}}

\newcommand{\postscript}[2]
 {\setlength{\epsfxsize}{#2\hsize}
  \centerline{\epsfbox{#1}}}

\renewcommand{\baselinestretch}{1.0}

\setlength{\oddsidemargin}{-0.365in}
\setlength{\evensidemargin}{-0.365in}
\setlength{\topmargin}{-0.3in}
\setlength{\headheight}{0in}
\setlength{\headsep}{0in}
\setlength{\textheight}{10.1in}
\setlength{\textwidth}{7in}
\makeatletter \renewenvironment{proof}[1][Proof] {\par\pushQED{\qed}\normalfont\topsep6\p@\@plus6\p@\relax\trivlist\item[\hskip\labelsep\bfseries#1\@addpunct{.}]\ignorespaces}{\popQED\endtrivlist\@endpefalse} \makeatother
\makeatletter
\renewenvironment{solution}[1][Solution] {\par\pushQED{\qed}\normalfont\topsep6\p@\@plus6\p@\relax\trivlist\item[\hskip\labelsep\bfseries#1\@addpunct{.}]\ignorespaces}{\popQED\endtrivlist\@endpefalse} \makeatother

\begin{document}
\noindent

%========================================================================
\noindent\framebox[\linewidth]{\shortstack[c]{
\Large{\textbf{Lab09-Network Flow}}\vspace{1mm}\\
CS214-Algorithm and Complexity, Xiaofeng Gao, Spring 2020.}}
\begin{center}
\footnotesize{\color{red}$*$ If there is any problem, please contact TA Shuodian Yu. }

\footnotesize{\color{blue}$*$ Name:Zehao Wang  \quad Student ID:518021910976 \quad Email: davidwang200099@sjtu.edu.cn}
\end{center}
\begin{enumerate}
\item Given a weighted directed graph $G(V, E)$ and its corresponding weight matrix $W=(w_{ij})_{n \times n}$ and shortest path matrix $D=(d_{ij})_{n \times n}$, where $w_{ij}$ is the weight of edge $(v_i, v_j)$ and $d_{ij}$ is the weight of a shortest path from pairwise vertex $v_i$ to $v_j$. Now, assume the weight of a particular edge $(v_a, v_b)$ is decreased from $w_{ab}$ to $w'_{ab}$. Design an algorithm to update matrix $D$ with respect to this change, whose time complexity should be no larger than $O(n^2)$. Describe your design first and write down your algorithm in the form of pseudo-code.
    \begin{solution}
        When $w_{ab}$ changes, for any different vertexes $u$ and $v$, the weight of path $u\rightarrow a \rightarrow b \rightarrow v$ changes. If this path has smaller weight than the original shortest path from $u$ to $v$, the smallest weight should be updated. Otherwise it should not.
        
        Both the outer loop and inner loop iterate $n$ times, and the update has $O(1)$ complexity. Therefore this algorithm has a complexity of $O(n^2)$.
        
        \begin{minipage}[t]{0.85\textwidth}
        	\begin{algorithm}[H]
        		%\algsetup{footnotesize}
        		%\scriptsize
        		\KwIn{Graph $G=(V,E)$, weight matrix $W$, shortest path matrix $D$, $a$, $b$, new weight of $(a,b)$: $w$.}
        		\KwOut{Updated shortest path matrix $D$}
        		\BlankLine
        		\caption{ShortestPaths(G,W,D,a,b,w)}
        		\label{Alg-1}
        		$D[a][b] \leftarrow w$\;
        		\For{$i=1$ to $|V|$}{
        		    \For{$j=1$ to $|V|$}{
        		        \If{$i \neq j$}{
        		            $D[i][j]=min\{D[i][j],D[i][a]+w+D[b][j]\}$\;
        	            }
        	        }
        	    }
        		\Return $D$\;
        	\end{algorithm}
        \end{minipage}
    \end{solution}

	\item Given a directed graph $G$, whose vertices and edges information are introduced in data file ``SCC.in''. Please find its number of Strongly Connected Components with respect to the following subquestions.
    \begin{enumerate}
    	\item Read the code and explanations of the provided C/C++ source code ``SCC.cpp'', and try to complete this implementation.
    	\item Visualize the above selected Strongly Connected Components for this graph $G$. Use the $Gephi$ or other software you preferred to draw the graph. {\color{blue}(If you feel that the data provided in ``SCC.in'' is not beautiful, you can also generate your own data with more vertices and edges than $G$ and draw an additional graph. Notice that results of your visualization will be taken into the consideration of Best Lab.)}

    \end{enumerate}
    \begin{solution}
    	\begin{enumerate}
    	
    	\item 
            Code has been put into the .zip file as SCC.cpp.
        \item 
            There are 666 SCCs in total.
        \begin{figure}[htbp]
        	
        	\centering
        	\includegraphics[width=0.5\textwidth]{SCC.png}
        	\caption{Visualization of SCC. Each vertex stands for one strongly-connected component.}
        	\label{SCC}
        \end{figure}
        \begin{figure}[htbp]
        	
        	\centering
        	\includegraphics[width=0.5\textwidth]{SCC-2.png}
        	\caption{Visualization of SCC. Numbers of vertexes in each SCC is marked as labels.}
        	\label{SCC-2}
        \end{figure}
    \end{enumerate}
    \end{solution}
	\item The \textbf{Minimum Cost Maximum Flow} problem (MCMF) is an optimization problem to find the cheapest possible way of sending the maximum amount of flow through a flow network. That is, in a flow network $G = (V, E)$ with a source $s\in V$ and a sink $t\in V$, where each edge $(u, v)\in E$ has a capacity $c(u,v) > 0$ and a cost $a(u,v) \ge 0$, find a maximum $s\text{-}t$ flow $f$ over all edges ($f(u, v) \ge 0)$, such that the total cost of $\sum_{(u, v) \in E} a(u, v) \cdot f(u, v)$ is minimized.

A common greedy approach to solve the MCMF problem can be described as follows: We can modify Ford-Fulkerson algorithm, where each time we choose the least cost path from $s$ to $t$. To do this correctly, when we add a back-edge to some edge $e$ into the residual graph, we give it a cost of $-a(e)$, representing that we get our money back if we undo the flow on it.

Note that such procedure may create a residual graph with negative-weight edges, which is not suitable for Dijkstra's Algorithm. However, motivated by Johnson's Algorithm, we can reweight the edge cost with vertex labels and convert the weight non-negative again.

Please prove the correctness of such greedy approach and implement this algorithm in C/C++. The file \emph{MCMF.in} is a test case, where the first line contains four graph parameters $n$, $m$, $s$, $t$, and the rest $m$ lines exhibit the information of $m$ edges. Each line contains four integers: $u_i$, $v_i$, $c_i$, $a_i$, denoting that there is an edge from $u_i$ to $v_i$ with capacity $c_i$ and cost $a_i$. {\color{blue}(Your source code should be named as \emph{MCMF.cpp} and output the maximum flow and minimum cost of this test case.)}

\fbox{
\begin{minipage}[t]{0.2\textwidth}
\textbf{Sample Input:}
	
	4 5 4 3 \\ 4 2 30 2 \\ 4 3 20 3 \\ 2 3 20 1 \\ 2 1 30 9 \\ 1 3 40 5
\end{minipage}
\begin{minipage}[t]{0.2\textwidth}
\textbf{Sample Output: }
	
	50 280
\end{minipage}}
\hspace{1cm}
\begin{minipage}[t]{0.45\textwidth}
\textbf{Remark:} The source code \emph{SCC.cpp}, and the input data \emph{SCC.in} and \emph{MCMF.in} are attached on the course webpage. Please include your .pdf, .tex, .cpp files for uploading with standard file names.
\end{minipage}
    \begin{solution}
    	\begin{enumerate}
    		\item 
        Assume that two flows $f$ and $f'$ are equal, but $f'$ costs less than $f$.
        
        Then consider the flow $f'-f$. 
        
        The flows into all vertexes except $s$ and $t$ equal the flows out of them. Therefore there is a cycle whose total cost is negative. Some flow can go around this cycle and pay less to go to the target.
        
        Therefore, as long as there is a negative cycle in a flow $f$, $f$ must not be the minimal cost flow.
        
        Assume that $f(i)$ denotes the cost found by this greedy approach when total flow from $s$ is $i$ and $f(i)$ is indeed the minimal cost.
        
        $f(0)$ is just the original graph. Obviously for a graph without negative cycle, $f(0)=0$.
        
        When the total flow is $i+1$, by the same greedy approach, we can also find $f(i+1)$.
        
        Assume that $f(i+1)$ is not the minimal cost and $f'(i+1)$ is the minimal cost.
        
        Then consider the flow $f'(i+1)-f(i)$.
        
        By definition, $f(i+1)-f(i)$ is a flow through the least cost path. However, $f'(i+1)-f(i)$ cost even less than $f(i+1)-f(i)$. That must be because there is a cycle with negative cycle in the graph, which means that $f(i)$ is not the minimal cost, which is contradictory to the former assumption. Therefore $f(i+1)$ must be the minimal cost.
        
        Therefore this greedy approach must be correct.
        \item For the test case, the maximal flow is 14098 and the minimal cost is 5290116.
    \end{enumerate}
    \end{solution}

\end{enumerate}






%========================================================================
\end{document}
