\documentclass[12pt,a4paper]{article}
\usepackage{ctex}
\usepackage{amsmath,amscd,amsbsy,amssymb,latexsym,url,bm,amsthm}
\usepackage{epsfig,graphicx,subfigure}
\usepackage{enumitem,balance}
\usepackage{wrapfig}
\usepackage{mathrsfs,euscript}
\usepackage[usenames]{xcolor}
\usepackage{hyperref}
\usepackage[vlined,ruled,linesnumbered]{algorithm2e}
\usepackage{array}
\hypersetup{colorlinks=true,linkcolor=black}

\newtheorem{theorem}{Theorem}
\newtheorem{lemma}[theorem]{Lemma}
\newtheorem{proposition}[theorem]{Proposition}
\newtheorem{corollary}[theorem]{Corollary}
\newtheorem{exercise}{Exercise}
\newtheorem*{solution}{Solution}
\newtheorem{definition}{Definition}
\theoremstyle{definition}

\renewcommand{\thefootnote}{\fnsymbol{footnote}}

\newcommand{\postscript}[2]
 {\setlength{\epsfxsize}{#2\hsize}
  \centerline{\epsfbox{#1}}}

\renewcommand{\baselinestretch}{1.0}

\setlength{\oddsidemargin}{-0.365in}
\setlength{\evensidemargin}{-0.365in}
\setlength{\topmargin}{-0.3in}
\setlength{\headheight}{0in}
\setlength{\headsep}{0in}
\setlength{\textheight}{10.1in}
\setlength{\textwidth}{7in}
\makeatletter \renewenvironment{proof}[1][Proof] {\par\pushQED{\qed}\normalfont\topsep6\p@\@plus6\p@\relax\trivlist\item[\hskip\labelsep\bfseries#1\@addpunct{.}]\ignorespaces}{\popQED\endtrivlist\@endpefalse} \makeatother
\makeatletter
\renewenvironment{solution}[1][Solution] {\par\pushQED{\qed}\normalfont\topsep6\p@\@plus6\p@\relax\trivlist\item[\hskip\labelsep\bfseries#1\@addpunct{.}]\ignorespaces}{\popQED\endtrivlist\@endpefalse} \makeatother

\begin{document}
\noindent

%========================================================================
\noindent\framebox[\linewidth]{\shortstack[c]{
\Large{\textbf{Lab07-Amortized Analysis}}\vspace{1mm}\\
CS214-Algorithm and Complexity, Xiaofeng Gao, Spring 2020.}}
\begin{center}
\footnotesize{\color{red}$*$ If there is any problem, please contact TA Shuodian Yu. }

\footnotesize{\color{blue}$*$ Name:Zehao Wang  \quad Student ID:518021910976 \quad Email: davidwang200099@sjtu.edu.cn}
\end{center}
\begin{enumerate}
	\item For the TABLE-DELETE Operation in Dynamic Tables, suppose we construct a table by multiplying its size by $\frac 23$ when the load factor drops below $\frac 13$. Using \emph{Potential Method} to prove that the amortized cost of a TABLE-DELETE that uses this strategy is bounded above by a constant.
	\begin{solution}
		Define 
		\begin{equation*}
		    \Phi_i=|2num_i-size_i|
		\end{equation*}
		as the potential function.
		
		If the $i$-th step is TABLE-DELETE, then
		\begin{itemize}
			\item 
			    When $\alpha_{i-1}>\frac{1}{2}$, then $num_i=num_{i-1}-1$ and $size_i=size_{i-1}$.
			    \begin{equation*}
			        \hat{C}_i=1+|2num_i-size_i|-|2num_{i-1}-size_{i-1}|
			    \end{equation*}
			    \begin{equation*}
			        =1+|2num_i-size_i|-|2num_{i-1}-size_{i-1}|
			    \end{equation*}
			    \begin{equation*}
			        =1+2num_i-size_i-2num_{i-1}+size_{i-1}=-1
			    \end{equation*}
			\item 
			    When $\alpha_{i-1}=\frac{1}{2}$, namely $num_{i-1}=\frac{size_{i-1}}{2}$, 
			    
			    then $num_i=num_{i-1}-1$ and $size_i=size_{i-1}$.
			    \begin{equation*}
			        \hat{C}_i=1+|2num_i-size_i|-|2num_{i-1}-size_{i-1}|
			    \end{equation*}
			    \begin{equation*}
			        =1+0+2(num_i+1)-size_i=3
			    \end{equation*}
			\item 
			    When $\frac{1}{3}<\alpha_{i-1}<\frac{1}{2}$, then $num_i=num_{i-1}-1$, $size_i=size_{i-1}$.
			    
			    \begin{equation}
			        \hat{C}_i=1+|2num_i-size_i|-|2num_{i-1}-size_{i-1}|
			    \end{equation}
			    \begin{equation*}
			        =1+size_i-2num_i+2num_{i-1}-size_{i-1}
			    \end{equation*}
			    \begin{equation*}
			        =1+2(num_{i-1}-num_i)
			    \end{equation*}
			    \begin{equation*}
			        =1+2=3
			    \end{equation*}
			\item 
			    When $\alpha_{i-1}=\frac{1}{3}$, then $num_i=num_{i-1}-1$, $size_i=\frac{2size_{i-1}}{3}$, $num_i+1=\frac{size_i}{2}$.
			    
			    \begin{equation*}
			        \hat{C}_i=num_{i-1}+|2num_i-size_i|-|2num_{i-1}-size_{i-1}|
			    \end{equation*}
			    \begin{equation*}
			        =num_{i-1}+1-|2num_{i-1}-3n_{i-1}|
			    \end{equation*}
			    \begin{equation*}
			        =num_{i-1}+1-num_{i-1}=1
			    \end{equation*}
		\end{itemize}
	
	    Therefore the amortized cost of TABLE-DELETE is bounded above by a constant.
	\end{solution}
	
	\item A \textbf{multistack} consists of an infinite series of stacks $S_0, S_1, S_2,\cdots$, where the $i^{th}$ stack $S_i$ can hold up to $3^i$ elements. Whenever a user attempts to push an element onto any full stack $S_i$, we first pop all the elements off $S_i$ and push them onto stack $S_{i+1}$ to make room. (Thus, if $S_{i+1}$ is already full, we first recursively move all its members to $S_{i+2}$ .) An illustrative example is shown in Figure \ref{Fig-MultiStack}. Moving a single element from one stack to the next takes $O(1)$ time. If we push a new element, \underline{we always intend to push it in stack $S_0$}.

	\begin{figure}[!htbp]
	\centering
	\includegraphics[width=0.5\textwidth]{Fig-MultiStack.pdf}
	\caption{An example of making room for one new element in a multistack.}
	\label{Fig-MultiStack}
	\end{figure}

    \begin{enumerate}
        \item In the worst case, how long does it take to push a new element onto a multistack containing $n$ elements?
        \item Prove that the amortized cost of a push operation is $O(\log n)$ by \emph{Aggregation Analysis}.
        \item {\color{red}(Optional Subquestion with Bonus)} Prove that the amortized cost of a push operation is $O(\log n)$ by \emph{Potential Method}.
    \end{enumerate}
    \begin{solution}
    	\ \\
    	\begin{enumerate}
    		\item
    		    The worst case happens when $n=\sum_{i=0}^{k} 3^i$. Every element in the multi-stack should be moved once.
    		    
    		    Then the time complexity of the push operation is $O(n)$.
    		\item 
    		    Assume that we try to push $\sum_{i=0}^{k+1}3^i+1$ elements into the multi-stack.
    		    
    		    %Then with the knowledge on geometry sequence, we can know $k=\log_3(2n-1)-2$.
    		    
    		    Consider the process of pushing elements to make the size of stack change from $\sum_{i=0}^{k}3^i+1$ to $\sum_{i=0}^{k+1}3^i+1$. In this process, $n=3^{k+1}$, namely $k=\log_3 n-1$.
    		    
    		    Assume when pushing the $m$-th element into the multi-stack, we need to move $T_m$ elements from one stack to another.
    		    
    		    By observation, we can see when $m$ is a multiple of $3^i$, $T_m=\sum_{j=0}^{i}3^j$. Otherwise,$T_m=1$.
    		    
    		    Therefore the total number of movement is:
    		    \begin{equation*}
    		        1+1+(3+1)+1+1+(3+1)+1+1+(9+3+1)+\cdots 
    		    \end{equation*}
    		    \begin{equation*}
    		        =3^0\times 3^{k+1}+3^1\times 3^{k}+3^2\times 3^{k-1}+\cdots+3^{k+1}\times 1
    		    \end{equation*}
    		    \begin{equation*}
    		        =(k+2)3^{k+1}
    		    \end{equation*}
    		    $3^{k+1}$ elements are pushed into the multi-stack  while changing the size of multi-stack change from $\sum_{i=0}^{k}+1=n-3^{k+1}$ to $\sum_{i=0}^{k+1}3^{i}+1$.
    		    
    		    We have
    		    \begin{equation*}
    		        \frac{(k+2)3^{k+1}}{3^{k+1}}=k+2=\log_3 n+1
    		    \end{equation*}
    		    
    		    Taking pushing new elements into consideration, the average number of operations is $\log_3 n+1+1$.
    		    
    		    Therefore the time complexity is $O(\log n)$.
    		    \item 
    		    Assume that the number of elements in the multi-stack is $n$.
    		    
    		    By observation, we can know that there exists an $k(n)$ such that $\sum_{i=0}^{k(n)}3^i<n\le \sum_{i=0}^{k(n)+1}3^i$.
    		    
    		    Then we can get $k(n)=\lceil \log_3(2n-1) \rceil-2$.
    		    
    		    And we define $|S_j(i)|$ to be the number of elements in the $j$-th stack when we have pushed $i$ elements into the multi-stack.($i\ge 0,\ j\ge 0$)
    		    
    		    Define
    		    \begin{equation*}
    		    \Phi(i)=
    		    \begin{cases}
    		    (k(i)\log_3 n -\sum_{j=0}^{k(i)} j|S_j(i)|&i\ne \sum_{j=0}^{m}3^j+1, integer\ m\\
    		    0&i=0
    		    \end{cases}
    		    \end{equation*}
    		    as potential function.
    		    
    		    Define 
    		    \begin{equation*}
    		        C_i=\sum_{j=0}^{k(i)}j|S_j(i)|-\sum_{j=0}^{k(i-1)}j|S_j(i-1)|
    		    \end{equation*}
    		    
    		    Then 
    		    \begin{equation*}
    		        \hat{C}_i=C_i+\Phi(i)-\Phi(i-1)
    		    \end{equation*}
    		    \begin{equation*}
    		    	=C_i+k(i)\log_3 i-\sum_{j=0}^{k(i)}j|S_j(i)|-k(i-1)\log_3(i-1)+\sum_{j=0}^{k(i-1)}j|S_j(i-1)|
    		    \end{equation*}
    		    \begin{equation*}
    		        \approx(k(i)-k(i-1))\log_3 i
    		    \end{equation*}
    		    \begin{equation*}
    		        \le log_3 i
    		    \end{equation*}
    		    
    		    Therefore we can know the time complexity of push is $O(\log n)$.
    	\end{enumerate}
    \end{solution}
	
	\item Given a graph $G = (V, E)$, and let $V'$ be a strict subset of $V$. Prove the following propositions.
	
	\begin{enumerate}
		\item Let $T$ be a minimum spanning tree of a $G$. Let $T'$ be the subgraph of $T$ induced by $V'$, and let $G'$ be the subgraph of $G$ induced by $V'$. Then $T'$ is a minimum spanning tree of $G'$ if $T'$ is connected.
		\item Let $e$ be a minimum weight edge which connects $V'$ and $V \setminus V'$. There exists a minimum weight spanning tree which contains e.
	\end{enumerate}
    \begin{solution}
    	\begin{enumerate}
    		\item 
    		\ \\
    		\begin{itemize}
    			\item 
    			    \textbf{Statement: If $T'$ is connected, it must be a tree.}  
    			    
    			    Proof:
    			    
    			    Assume $T'$ is connected but $T'$ is not a tree. According to the definition of a tree, if $T'$ is connected but not a tree, it must have a cycle in it.
    			    
    			    However, $T=T'\cup (T\setminus T')$. If $T'$ has a cycle, $T$ must have a cycle, which is contradictory to the condition that $T$ is a tree. Therefore $T'$ must be a tree if it is connected.
    			\item
    			    Assume $T$ is a minimum spanning tree but $T'$ is not a minimum spanning tree, then there is a certain edge $e_0$ which connect two vertexes in $V'$ and form a cycle $C_0$ with certain edges in $E$. What's more, $w(e_0)<w(a)$ for an edge $a_0\in C_0$.
    			    
    			    In this case, the total weight of $T'\oplus(a_0,e_0)$ is smaller than $T'$. However, because $T'\subseteq T$, therefore $(T'\oplus(a_0,e_0))\cup (T \setminus T') $ is a spanning tree with a smaller weight than $T$, which is contradictory to the condition that $T$ is a minimal spanning tree.
    			    
    			    Therefore the assumption does not make sense. Then $T'$ is a minimum spanning tree of $G'$ if $T'$ is connected.
    			
    			    
    		\end{itemize}
    		\item 
    		    Assume that the minimum spanning tree $T'$ does not contain $e$.
    		    
    		    Edges in the minimum spanning tree $T'$ can be divided into three parts:
    		    \begin{itemize}
    		    	\item Edges connecting vertexes in $V'$
    		    	\item Edges connecting vertexes in $V \setminus V'$
    		    	\item A single edge $e'$ which connects one vertex in $V$ and another in $V \setminus V'$
    		    \end{itemize}    
    	        Because the two vertexes connected by $e'$ and $e$ are in $V$ and $V \setminus V'$ separately, neither $e'$ nor $e$ can form any circle with edges in $E'$ or edges connecting vertexes in $V \setminus V'$. 
    	        
    	        According to the definition of $e$, its weight is smaller than $e'$.
    	        
    	        Then replace $e'$ with $e$ and we can still get a spanning tree $T$. This new tree has a smaller weight than $T'$, which is contradictory to the assumption that $T'$ is minimum spanning tree.
    	        
    	        Therefore the assumption does not make sense. There must exist a minimum weight spanning tree which contains e.
    	\end{enumerate}
    \end{solution}
\end{enumerate}



\textbf{Remark:} Please include your .pdf, .tex files for uploading with standard file names.


%========================================================================
\end{document}
